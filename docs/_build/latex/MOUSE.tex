%% Generated by Sphinx.
\def\sphinxdocclass{report}
\documentclass[letterpaper,10pt,english]{sphinxmanual}
\ifdefined\pdfpxdimen
   \let\sphinxpxdimen\pdfpxdimen\else\newdimen\sphinxpxdimen
\fi \sphinxpxdimen=49336sp\relax

\usepackage[margin=1in,marginparwidth=0.5in]{geometry}
\usepackage[utf8]{inputenc}
\ifdefined\DeclareUnicodeCharacter
  \DeclareUnicodeCharacter{00A0}{\nobreakspace}
\fi
\usepackage{cmap}
\usepackage[T1]{fontenc}
\usepackage{amsmath,amssymb,amstext}
\usepackage{babel}
\usepackage{times}
\usepackage[Bjarne]{fncychap}
\usepackage{longtable}
\usepackage{sphinx}

\usepackage{multirow}
\usepackage{eqparbox}

% Include hyperref last.
\usepackage{hyperref}
% Fix anchor placement for figures with captions.
\usepackage{hypcap}% it must be loaded after hyperref.
% Set up styles of URL: it should be placed after hyperref.
\urlstyle{same}
\addto\captionsenglish{\renewcommand{\contentsname}{Contents:}}

\addto\captionsenglish{\renewcommand{\figurename}{Fig.\@ }}
\addto\captionsenglish{\renewcommand{\tablename}{Table }}
\addto\captionsenglish{\renewcommand{\literalblockname}{Listing }}

\addto\extrasenglish{\def\pageautorefname{page}}

\setcounter{tocdepth}{2}



\title{MOUSE Documentation}
\date{Dec 31, 2016}
\release{0.3}
\author{M. Yetisir}
\newcommand{\sphinxlogo}{}
\renewcommand{\releasename}{Release}
\makeindex

\begin{document}

\maketitle
\sphinxtableofcontents
\phantomsection\label{\detokenize{index::doc}}



\chapter{User Manual}
\label{\detokenize{UserManual:user-manual}}\label{\detokenize{UserManual::doc}}\label{\detokenize{UserManual:welcome-to-mouse-s-documentation}}

\section{Overview}
\label{\detokenize{UserManual:overview}}

\section{Installation Guide}
\label{\detokenize{UserManual:installation-guide}}

\subsection{Installing python interpreter and dependencies}
\label{\detokenize{UserManual:installing-python-interpreter-and-dependencies}}

\subsection{Installing MOUSE}
\label{\detokenize{UserManual:installing-mouse}}

\section{Basic Usage}
\label{\detokenize{UserManual:basic-usage}}

\section{Basic Settings}
\label{\detokenize{UserManual:basic-settings}}

\chapter{MOUSE Reference Manual}
\label{\detokenize{MouseReferenceManual::doc}}\label{\detokenize{MouseReferenceManual:mouse-reference-manual}}

\section{Overview}
\label{\detokenize{MouseReferenceManual:overview}}

\section{MOUSE Usage}
\label{\detokenize{MouseReferenceManual:mouse-usage}}

MOUSE: An Up-Scaling Utility for DEM Simulations


\begin{sphinxVerbatim}[commandchars=\\\{\}]
\PYG{n}{usage}\PYG{p}{:} \PYG{n}{MOUSE} \PYG{p}{[}\PYG{o}{\PYGZhy{}}\PYG{n}{h}\PYG{p}{]} \PYG{p}{\PYGZob{}}\PYG{n}{UDEC}\PYG{p}{,}\PYG{n}{HODS}\PYG{p}{,}\PYG{n}{OSTRICH}\PYG{p}{,}\PYG{n}{ABAQUS}\PYG{p}{\PYGZcb{}} \PYG{o}{.}\PYG{o}{.}\PYG{o}{.}
\end{sphinxVerbatim}
\begin{description}
\item[{Sub-commands:}] \leavevmode\begin{description}
\item[{\sphinxstylestrong{UDEC}}] \leavevmode
Undocumented

\begin{sphinxVerbatim}[commandchars=\\\{\}]
\PYG{n}{usage}\PYG{p}{:} \PYG{n}{MOUSE} \PYG{n}{UDEC} \PYG{p}{[}\PYG{o}{\PYGZhy{}}\PYG{n}{h}\PYG{p}{]} \PYG{o}{\PYGZhy{}}\PYG{n}{n} \PYG{n}{NAME}
\end{sphinxVerbatim}
\begin{description}
\item[{Options:}] \leavevmode\begin{optionlist}{3cm}
\item [-n, -{-}name]  
Name of the file containing the model data without the extension
\end{optionlist}

\end{description}

\item[{\sphinxstylestrong{HODS}}] \leavevmode
Undocumented

\begin{sphinxVerbatim}[commandchars=\\\{\}]
\PYG{n}{usage}\PYG{p}{:} \PYG{n}{MOUSE} \PYG{n}{HODS} \PYG{p}{[}\PYG{o}{\PYGZhy{}}\PYG{n}{h}\PYG{p}{]} \PYG{o}{\PYGZhy{}}\PYG{n}{n} \PYG{n}{NAME} \PYG{p}{[}\PYG{o}{\PYGZhy{}}\PYG{n}{x} \PYG{n}{REVX}\PYG{p}{]} \PYG{p}{[}\PYG{o}{\PYGZhy{}}\PYG{n}{y} \PYG{n}{REVY}\PYG{p}{]} \PYG{p}{[}\PYG{o}{\PYGZhy{}}\PYG{n}{r} \PYG{n}{REVRADIUS}\PYG{p}{]}
\end{sphinxVerbatim}
\begin{description}
\item[{Options:}] \leavevmode\begin{optionlist}{3cm}
\item [-n, -{-}name]  
Name of the file containing the model data without the extension
\item [-x, -{-}revX]  
x coordinate of REV centre
\item [-y, -{-}revY]  
y coordinate of REV centre
\item [-r, -{-}revRadius]  
Radius of REV centre
\end{optionlist}

\end{description}

\item[{\sphinxstylestrong{OSTRICH}}] \leavevmode
Undocumented

\begin{sphinxVerbatim}[commandchars=\\\{\}]
\PYG{n}{usage}\PYG{p}{:} \PYG{n}{MOUSE} \PYG{n}{OSTRICH} \PYG{p}{[}\PYG{o}{\PYGZhy{}}\PYG{n}{h}\PYG{p}{]} \PYG{o}{\PYGZhy{}}\PYG{n}{n} \PYG{n}{NAME} \PYG{p}{[}\PYG{o}{\PYGZhy{}}\PYG{n+nb}{id} \PYG{n}{IDENTITY}\PYG{p}{]} \PYG{p}{[}\PYG{o}{\PYGZhy{}}\PYG{n}{p} \PYG{n}{PARALLEL}\PYG{p}{]} \PYG{p}{[}\PYG{o}{\PYGZhy{}}\PYG{n}{c} \PYG{n}{CORES}\PYG{p}{]}
                     \PYG{p}{[}\PYG{o}{\PYGZhy{}}\PYG{n}{o} \PYG{n}{OPTIMIZER}\PYG{p}{]}
\end{sphinxVerbatim}
\begin{description}
\item[{Options:}] \leavevmode\begin{optionlist}{3cm}
\item [-n, -{-}name]  
Name of the file containing the model data without the extension
\item [-id, -{-}identity]  
identification Number
\item [-p=True, -{-}parallel=True]  
use parallel processing
\item [-c=4, -{-}cores=4]  
number of logical cores to use for parallel processing
\item [-o=ParticleSwarm, -{-}optimizer=ParticleSwarm]  
optimization algorithm
\end{optionlist}

\end{description}

\item[{\sphinxstylestrong{ABAQUS}}] \leavevmode
Undocumented

\begin{sphinxVerbatim}[commandchars=\\\{\}]
\PYG{n}{usage}\PYG{p}{:} \PYG{n}{MOUSE} \PYG{n}{ABAQUS} \PYG{p}{[}\PYG{o}{\PYGZhy{}}\PYG{n}{h}\PYG{p}{]} \PYG{o}{\PYGZhy{}}\PYG{n}{n} \PYG{n}{NAME}
\end{sphinxVerbatim}
\begin{description}
\item[{Options:}] \leavevmode\begin{optionlist}{3cm}
\item [-n, -{-}name]  
Name of the file containing the model data without the extension
\end{optionlist}

\end{description}

\end{description}

\end{description}


\section{MOUSE Documentation}
\label{\detokenize{MouseReferenceManual:mouse-documentation}}\label{\detokenize{MouseReferenceManual:module-MOUSE}}\index{MOUSE (module)}\index{SplashScreen (class in MOUSE)}

\begin{fulllineitems}
\phantomsection\label{\detokenize{MouseReferenceManual:MOUSE.SplashScreen}}\pysiglinewithargsret{\sphinxstrong{class }\sphinxcode{MOUSE.}\sphinxbfcode{SplashScreen}}{\emph{boxWidth=55}, \emph{textWidth=70}, \emph{padWidth=15}}{}
Bases: \sphinxcode{object}

Creates the splash screen and interface for MOUSE

This class allows for the generation of an introduction screen for MOUSE. Here, a collection of printing methods are created in order to provide an environment for creating a consistent splash screen and interface.
\index{boxWidth (MOUSE.SplashScreen attribute)}

\begin{fulllineitems}
\phantomsection\label{\detokenize{MouseReferenceManual:MOUSE.SplashScreen.boxWidth}}\pysigline{\sphinxbfcode{boxWidth}}
\sphinxstyleemphasis{int} -- Character width of text box for splash screen

\end{fulllineitems}

\index{textWidth (MOUSE.SplashScreen attribute)}

\begin{fulllineitems}
\phantomsection\label{\detokenize{MouseReferenceManual:MOUSE.SplashScreen.textWidth}}\pysigline{\sphinxbfcode{textWidth}}
\sphinxstyleemphasis{int} -- Character width of text area for splash screen

\end{fulllineitems}

\index{padWidth (MOUSE.SplashScreen attribute)}

\begin{fulllineitems}
\phantomsection\label{\detokenize{MouseReferenceManual:MOUSE.SplashScreen.padWidth}}\pysigline{\sphinxbfcode{padWidth}}
\sphinxstyleemphasis{int} -- Character width of text area for padding on splash screen

\end{fulllineitems}

\index{printBoxLine() (MOUSE.SplashScreen method)}

\begin{fulllineitems}
\phantomsection\label{\detokenize{MouseReferenceManual:MOUSE.SplashScreen.printBoxLine}}\pysiglinewithargsret{\sphinxbfcode{printBoxLine}}{}{}
Prints a horizontal line for the box in the centre of the console
\begin{quote}\begin{description}
\item[{Returns}] \leavevmode
Prints a centred horizontal dashed line of length self.boxWidth on the console

\item[{Return type}] \leavevmode
None

\end{description}\end{quote}

\end{fulllineitems}

\index{printCentre() (MOUSE.SplashScreen method)}

\begin{fulllineitems}
\phantomsection\label{\detokenize{MouseReferenceManual:MOUSE.SplashScreen.printCentre}}\pysiglinewithargsret{\sphinxbfcode{printCentre}}{\emph{text}}{}
Prints text in the centre of the console
\begin{quote}\begin{description}
\item[{Parameters}] \leavevmode
\sphinxstyleliteralstrong{text} (\sphinxstyleliteralemphasis{str}) -- text to be printed in the centre of the console

\item[{Returns}] \leavevmode
Prints str to the centre of the console

\item[{Return type}] \leavevmode
None

\end{description}\end{quote}

\end{fulllineitems}

\index{printFullLine() (MOUSE.SplashScreen method)}

\begin{fulllineitems}
\phantomsection\label{\detokenize{MouseReferenceManual:MOUSE.SplashScreen.printFullLine}}\pysiglinewithargsret{\sphinxbfcode{printFullLine}}{}{}
Prints a horizontal line across the text width of the console
\begin{quote}\begin{description}
\item[{Returns}] \leavevmode
Prints a horizontal dashed line of length self.textWidth on the console

\item[{Return type}] \leavevmode
None

\end{description}\end{quote}

\end{fulllineitems}

\index{printInBox() (MOUSE.SplashScreen method)}

\begin{fulllineitems}
\phantomsection\label{\detokenize{MouseReferenceManual:MOUSE.SplashScreen.printInBox}}\pysiglinewithargsret{\sphinxbfcode{printInBox}}{\emph{text}}{}
Prints text in the centre of the box
\begin{quote}\begin{description}
\item[{Parameters}] \leavevmode
\sphinxstyleliteralstrong{text} (\sphinxstyleliteralemphasis{str}) -- text to be printed in the centre of the box

\item[{Returns}] \leavevmode
Prints a horizontal dashed line of length self.textWidth on the console

\item[{Return type}] \leavevmode
None

\end{description}\end{quote}

\end{fulllineitems}

\index{printModule() (MOUSE.SplashScreen method)}

\begin{fulllineitems}
\phantomsection\label{\detokenize{MouseReferenceManual:MOUSE.SplashScreen.printModule}}\pysiglinewithargsret{\sphinxbfcode{printModule}}{\emph{module}, \emph{status}}{}
Prints module and installation status in the splash box
\begin{quote}\begin{description}
\item[{Parameters}] \leavevmode\begin{itemize}
\item {} 
\sphinxstyleliteralstrong{module} (\sphinxstyleliteralemphasis{str}) -- name of the module

\item {} 
\sphinxstyleliteralstrong{status} (\sphinxstyleliteralemphasis{str}) -- module status {[}installed, available, unavailable{]}

\end{itemize}

\item[{Returns}] \leavevmode
Prints the module name with the status in the splash box

\item[{Return type}] \leavevmode
None

\end{description}\end{quote}

\end{fulllineitems}

\index{printSplash() (MOUSE.SplashScreen method)}

\begin{fulllineitems}
\phantomsection\label{\detokenize{MouseReferenceManual:MOUSE.SplashScreen.printSplash}}\pysiglinewithargsret{\sphinxbfcode{printSplash}}{}{}
Clears the console and prints the splash screen to the console
\begin{quote}\begin{description}
\item[{Returns}] \leavevmode
Splash screen printed on console

\item[{Return type}] \leavevmode
None

\end{description}\end{quote}

\begin{sphinxadmonition}{note}{Todo}

Import Modules and stuses from module files rather than hard coding them into this method
\end{sphinxadmonition}

\end{fulllineitems}


\end{fulllineitems}

\index{createParser() (in module MOUSE)}

\begin{fulllineitems}
\phantomsection\label{\detokenize{MouseReferenceManual:MOUSE.createParser}}\pysiglinewithargsret{\sphinxcode{MOUSE.}\sphinxbfcode{createParser}}{}{}
Creates an argparse parser object for MOUSE and imports argparse subparsers for each MOUSE Module

\begin{sphinxadmonition}{note}{Todo}

Scan subparsers from module files and import in order to remove hard-coded dependance
\end{sphinxadmonition}

\begin{sphinxadmonition}{note}{Note:}
Currenlty subparser imports are hard-coded in
\end{sphinxadmonition}
\begin{quote}\begin{description}
\item[{Returns}] \leavevmode
the main argument parser for MOUSE populated with all required subparsers form modules.

\item[{Return type}] \leavevmode
argparse.ArgumentParser

\end{description}\end{quote}

\end{fulllineitems}



\section{Modules Documentation}
\label{\detokenize{MouseReferenceManual:modules-documentation}}

\subsection{Modules.Module\_ABAQUS module}
\label{\detokenize{MouseReferenceManual:module-Modules.Module_ABAQUS}}\label{\detokenize{MouseReferenceManual:modules-module-abaqus-module}}\index{Modules.Module\_ABAQUS (module)}\index{Module\_ABAQUS (class in Modules.Module\_ABAQUS)}

\begin{fulllineitems}
\phantomsection\label{\detokenize{MouseReferenceManual:Modules.Module_ABAQUS.Module_ABAQUS}}\pysiglinewithargsret{\sphinxstrong{class }\sphinxcode{Modules.Module\_ABAQUS.}\sphinxbfcode{Module\_ABAQUS}}{\emph{baseName}}{}
Bases: {\hyperref[\detokenize{MouseReferenceManual:Modules.Base.ContinuumModuleBaseClass}]{\sphinxcrossref{\sphinxcode{Modules.Base.ContinuumModuleBaseClass}}}}
\index{formatOutput() (Modules.Module\_ABAQUS.Module\_ABAQUS method)}

\begin{fulllineitems}
\phantomsection\label{\detokenize{MouseReferenceManual:Modules.Module_ABAQUS.Module_ABAQUS.formatOutput}}\pysiglinewithargsret{\sphinxbfcode{formatOutput}}{}{}
Formats ABAQUS data into consistent nested lists and writes them to binary file
\begin{quote}\begin{description}
\item[{Returns}] \leavevmode
writes serialized binary data to file

\item[{Return type}] \leavevmode
None

\end{description}\end{quote}

\end{fulllineitems}

\index{parseInput() (Modules.Module\_ABAQUS.Module\_ABAQUS method)}

\begin{fulllineitems}
\phantomsection\label{\detokenize{MouseReferenceManual:Modules.Module_ABAQUS.Module_ABAQUS.parseInput}}\pysiglinewithargsret{\sphinxbfcode{parseInput}}{}{}
Parses input file
\begin{quote}\begin{description}
\item[{Returns}] \leavevmode
returns data in a structured array

\item[{Return type}] \leavevmode
struct

\end{description}\end{quote}

\end{fulllineitems}

\index{run() (Modules.Module\_ABAQUS.Module\_ABAQUS method)}

\begin{fulllineitems}
\phantomsection\label{\detokenize{MouseReferenceManual:Modules.Module_ABAQUS.Module_ABAQUS.run}}\pysiglinewithargsret{\sphinxbfcode{run}}{}{}
runs the HODS homogenization Module which creates input files for OSTRICH MOUSE Module
\begin{quote}\begin{description}
\item[{Returns}] \leavevmode
MOUSE homogenization data files

\item[{Return type}] \leavevmode
None

\end{description}\end{quote}

\end{fulllineitems}

\index{setParameters() (Modules.Module\_ABAQUS.Module\_ABAQUS method)}

\begin{fulllineitems}
\phantomsection\label{\detokenize{MouseReferenceManual:Modules.Module_ABAQUS.Module_ABAQUS.setParameters}}\pysiglinewithargsret{\sphinxbfcode{setParameters}}{\emph{revCentreX=None}, \emph{revCentreY=None}, \emph{revRadius=None}}{}
Sets module parameters

\begin{sphinxadmonition}{note}{Todo}

assess revCentreX, revCentreY and revRadius from data rather than from input file
\end{sphinxadmonition}
\begin{quote}\begin{description}
\item[{Parameters}] \leavevmode
\sphinxstyleliteralstrong{parameters} (\sphinxstyleliteralemphasis{dict}) -- new parameters to be set

\item[{Returns}] \leavevmode
Sets module parameters

\item[{Return type}] \leavevmode
None

\end{description}\end{quote}

\end{fulllineitems}


\end{fulllineitems}

\index{importModelData() (in module Modules.Module\_ABAQUS)}

\begin{fulllineitems}
\phantomsection\label{\detokenize{MouseReferenceManual:Modules.Module_ABAQUS.importModelData}}\pysiglinewithargsret{\sphinxcode{Modules.Module\_ABAQUS.}\sphinxbfcode{importModelData}}{\emph{modelName}}{}
Imports the input model parameters and assigns them to a global modelData variable
\begin{quote}\begin{description}
\item[{Parameters}] \leavevmode
\sphinxstyleliteralstrong{modelName} (\sphinxstyleliteralemphasis{str}) -- Name of file containing the model data.

\item[{Returns}] \leavevmode
Assigns model parameters from file to global modelData

\item[{Return type}] \leavevmode
None

\end{description}\end{quote}

\end{fulllineitems}

\index{parserHandler() (in module Modules.Module\_ABAQUS)}

\begin{fulllineitems}
\phantomsection\label{\detokenize{MouseReferenceManual:Modules.Module_ABAQUS.parserHandler}}\pysiglinewithargsret{\sphinxcode{Modules.Module\_ABAQUS.}\sphinxbfcode{parserHandler}}{\emph{args}}{}
Function called after argparse subparser is executed
\begin{quote}\begin{description}
\item[{Parameters}] \leavevmode
\sphinxstyleliteralstrong{args} (\sphinxstyleliteralemphasis{argparse.Arguments}) -- arparse parsed command line arguments.

\item[{Returns}] \leavevmode
initializes ABAQUS Module and runs it

\item[{Return type}] \leavevmode
None

\end{description}\end{quote}

\end{fulllineitems}

\index{populateArgumentParser() (in module Modules.Module\_ABAQUS)}

\begin{fulllineitems}
\phantomsection\label{\detokenize{MouseReferenceManual:Modules.Module_ABAQUS.populateArgumentParser}}\pysiglinewithargsret{\sphinxcode{Modules.Module\_ABAQUS.}\sphinxbfcode{populateArgumentParser}}{\emph{parser}}{}
Adds arguments to the argument parser
\begin{quote}\begin{description}
\item[{Parameters}] \leavevmode
\sphinxstyleliteralstrong{parser} (\sphinxstyleliteralemphasis{argparse.ArgumentParser}) -- empty argparse subparser

\item[{Returns}] \leavevmode
same argparse supparser, now populated with arguments

\item[{Return type}] \leavevmode
argparse.ArgumentParser

\end{description}\end{quote}

\end{fulllineitems}



\subsection{Modules.Module\_HODS module}
\label{\detokenize{MouseReferenceManual:modules-module-hods-module}}\label{\detokenize{MouseReferenceManual:module-Modules.Module_HODS}}\index{Modules.Module\_HODS (module)}\index{Module\_HODS (class in Modules.Module\_HODS)}

\begin{fulllineitems}
\phantomsection\label{\detokenize{MouseReferenceManual:Modules.Module_HODS.Module_HODS}}\pysiglinewithargsret{\sphinxstrong{class }\sphinxcode{Modules.Module\_HODS.}\sphinxbfcode{Module\_HODS}}{\emph{baseName}}{}
Bases: {\hyperref[\detokenize{MouseReferenceManual:Modules.Base.HomogenizationModuleBaseClass}]{\sphinxcrossref{\sphinxcode{Modules.Base.HomogenizationModuleBaseClass}}}}

Creates the HODS model interface for MOUSE

This class allows for the generation of the usage of HODS through the MOUSE framework. Because HODS was also written in python, a direct link can be established between the programs rather than relying on I/O protols.
\index{stressHistory (Modules.Module\_HODS.Module\_HODS attribute)}

\begin{fulllineitems}
\phantomsection\label{\detokenize{MouseReferenceManual:Modules.Module_HODS.Module_HODS.stressHistory}}\pysigline{\sphinxbfcode{stressHistory}}
\sphinxstyleemphasis{nested list of float} -- List of homogenized stress tensor history

\end{fulllineitems}

\index{strainHistory (Modules.Module\_HODS.Module\_HODS attribute)}

\begin{fulllineitems}
\phantomsection\label{\detokenize{MouseReferenceManual:Modules.Module_HODS.Module_HODS.strainHistory}}\pysigline{\sphinxbfcode{strainHistory}}
\sphinxstyleemphasis{nested list of float} -- List of homogenized strain tensor history

\end{fulllineitems}

\index{timeHistory (Modules.Module\_HODS.Module\_HODS attribute)}

\begin{fulllineitems}
\phantomsection\label{\detokenize{MouseReferenceManual:Modules.Module_HODS.Module_HODS.timeHistory}}\pysigline{\sphinxbfcode{timeHistory}}
\sphinxstyleemphasis{list of float} -- List of simulation time steps

\end{fulllineitems}

\index{revCentreX (Modules.Module\_HODS.Module\_HODS attribute)}

\begin{fulllineitems}
\phantomsection\label{\detokenize{MouseReferenceManual:Modules.Module_HODS.Module_HODS.revCentreX}}\pysigline{\sphinxbfcode{revCentreX}}
\sphinxstyleemphasis{float} -- x position of centre of REV

\end{fulllineitems}

\index{revCentreY (Modules.Module\_HODS.Module\_HODS attribute)}

\begin{fulllineitems}
\phantomsection\label{\detokenize{MouseReferenceManual:Modules.Module_HODS.Module_HODS.revCentreY}}\pysigline{\sphinxbfcode{revCentreY}}
\sphinxstyleemphasis{float} -- y position of centre of REV

\end{fulllineitems}

\index{revRadius (Modules.Module\_HODS.Module\_HODS attribute)}

\begin{fulllineitems}
\phantomsection\label{\detokenize{MouseReferenceManual:Modules.Module_HODS.Module_HODS.revRadius}}\pysigline{\sphinxbfcode{revRadius}}
\sphinxstyleemphasis{float} -- radius of REV

\end{fulllineitems}

\index{formatOutput() (Modules.Module\_HODS.Module\_HODS method)}

\begin{fulllineitems}
\phantomsection\label{\detokenize{MouseReferenceManual:Modules.Module_HODS.Module_HODS.formatOutput}}\pysiglinewithargsret{\sphinxbfcode{formatOutput}}{}{}
Formats Homogenization data into consistent nested lists and writes them to binary file
\begin{quote}\begin{description}
\item[{Returns}] \leavevmode
writes serialized binary data to file

\item[{Return type}] \leavevmode
None

\end{description}\end{quote}

\end{fulllineitems}

\index{parseInput() (Modules.Module\_HODS.Module\_HODS method)}

\begin{fulllineitems}
\phantomsection\label{\detokenize{MouseReferenceManual:Modules.Module_HODS.Module_HODS.parseInput}}\pysiglinewithargsret{\sphinxbfcode{parseInput}}{}{}
Parses input file
\begin{quote}\begin{description}
\item[{Returns}] \leavevmode
returns data in a structured array

\item[{Return type}] \leavevmode
struct

\end{description}\end{quote}

\end{fulllineitems}

\index{run() (Modules.Module\_HODS.Module\_HODS method)}

\begin{fulllineitems}
\phantomsection\label{\detokenize{MouseReferenceManual:Modules.Module_HODS.Module_HODS.run}}\pysiglinewithargsret{\sphinxbfcode{run}}{}{}
runs the HODS homogenization Module which creates input files for OSTRICH MOUSE Module
\begin{quote}\begin{description}
\item[{Returns}] \leavevmode
MOUSE homogenization data files

\item[{Return type}] \leavevmode
None

\end{description}\end{quote}

\end{fulllineitems}

\index{setParameters() (Modules.Module\_HODS.Module\_HODS method)}

\begin{fulllineitems}
\phantomsection\label{\detokenize{MouseReferenceManual:Modules.Module_HODS.Module_HODS.setParameters}}\pysiglinewithargsret{\sphinxbfcode{setParameters}}{\emph{args}}{}
Sets module parameters

\begin{sphinxadmonition}{note}{Todo}

assess revCentreX, revCentreY and revRadius from data rather than from input file
\end{sphinxadmonition}
\begin{quote}\begin{description}
\item[{Parameters}] \leavevmode
\sphinxstyleliteralstrong{parameters} (\sphinxstyleliteralemphasis{dict}) -- new parameters to be set

\item[{Returns}] \leavevmode
Sets module parameters

\item[{Return type}] \leavevmode
None

\end{description}\end{quote}

\end{fulllineitems}


\end{fulllineitems}

\index{importModelData() (in module Modules.Module\_HODS)}

\begin{fulllineitems}
\phantomsection\label{\detokenize{MouseReferenceManual:Modules.Module_HODS.importModelData}}\pysiglinewithargsret{\sphinxcode{Modules.Module\_HODS.}\sphinxbfcode{importModelData}}{\emph{modelName}}{}
Imports the input model parameters and assigns them to a global modelData variable
\begin{quote}\begin{description}
\item[{Parameters}] \leavevmode
\sphinxstyleliteralstrong{modelName} (\sphinxstyleliteralemphasis{str}) -- Name of file containing the model data.

\item[{Returns}] \leavevmode
Assigns model parameters from file to global modelData

\item[{Return type}] \leavevmode
None

\end{description}\end{quote}

\end{fulllineitems}

\index{parserHandler() (in module Modules.Module\_HODS)}

\begin{fulllineitems}
\phantomsection\label{\detokenize{MouseReferenceManual:Modules.Module_HODS.parserHandler}}\pysiglinewithargsret{\sphinxcode{Modules.Module\_HODS.}\sphinxbfcode{parserHandler}}{\emph{args}}{}
Function called after argparse subparser is executed
\begin{quote}\begin{description}
\item[{Parameters}] \leavevmode
\sphinxstyleliteralstrong{args} (\sphinxstyleliteralemphasis{argparse.Arguments}) -- arparse parsed command line arguments.

\item[{Returns}] \leavevmode
initializes HODS Module and runs it

\item[{Return type}] \leavevmode
None

\end{description}\end{quote}

\end{fulllineitems}

\index{populateArgumentParser() (in module Modules.Module\_HODS)}

\begin{fulllineitems}
\phantomsection\label{\detokenize{MouseReferenceManual:Modules.Module_HODS.populateArgumentParser}}\pysiglinewithargsret{\sphinxcode{Modules.Module\_HODS.}\sphinxbfcode{populateArgumentParser}}{\emph{parser}}{}
Adds arguments to the argument parser
\begin{quote}\begin{description}
\item[{Parameters}] \leavevmode
\sphinxstyleliteralstrong{parser} (\sphinxstyleliteralemphasis{argparse.ArgumentParser}) -- empty argparse subparser

\item[{Returns}] \leavevmode
same argparse supparser, now populated with arguments

\item[{Return type}] \leavevmode
argparse.ArgumentParser

\end{description}\end{quote}

\end{fulllineitems}



\subsection{Modules.Module\_OSTRICH module}
\label{\detokenize{MouseReferenceManual:modules-module-ostrich-module}}\label{\detokenize{MouseReferenceManual:module-Modules.Module_OSTRICH}}\index{Modules.Module\_OSTRICH (module)}\index{Module\_OSTRICH (class in Modules.Module\_OSTRICH)}

\begin{fulllineitems}
\phantomsection\label{\detokenize{MouseReferenceManual:Modules.Module_OSTRICH.Module_OSTRICH}}\pysiglinewithargsret{\sphinxstrong{class }\sphinxcode{Modules.Module\_OSTRICH.}\sphinxbfcode{Module\_OSTRICH}}{\emph{baseName}}{}
Bases: {\hyperref[\detokenize{MouseReferenceManual:Modules.Base.ParameterEstimationModuleBaseClass}]{\sphinxcrossref{\sphinxcode{Modules.Base.ParameterEstimationModuleBaseClass}}}}

Creates the OSTRICH interface for MOUSE

This class allows for the generation of the usage of UDEC through the MOUSE framework. As of current, there is still no capacity for full automation due to UDEC API limitations. As such, a batch UDEC script is generated which can then be called in UDEC with one command.
\index{identity (Modules.Module\_OSTRICH.Module\_OSTRICH attribute)}

\begin{fulllineitems}
\phantomsection\label{\detokenize{MouseReferenceManual:Modules.Module_OSTRICH.Module_OSTRICH.identity}}\pysigline{\sphinxbfcode{identity}}
\sphinxstyleemphasis{int} -- for repeat trials, a different identity can be assigned to each game

\end{fulllineitems}

\index{MPI (Modules.Module\_OSTRICH.Module\_OSTRICH attribute)}

\begin{fulllineitems}
\phantomsection\label{\detokenize{MouseReferenceManual:Modules.Module_OSTRICH.Module_OSTRICH.MPI}}\pysigline{\sphinxbfcode{MPI}}
\sphinxstyleemphasis{bool} -- OSTRICH uses parallel processing if True

\end{fulllineitems}

\index{cores (Modules.Module\_OSTRICH.Module\_OSTRICH attribute)}

\begin{fulllineitems}
\phantomsection\label{\detokenize{MouseReferenceManual:Modules.Module_OSTRICH.Module_OSTRICH.cores}}\pysigline{\sphinxbfcode{cores}}
\sphinxstyleemphasis{int} -- number of logical cores to be used for optimization

\end{fulllineitems}

\index{optimizer (Modules.Module\_OSTRICH.Module\_OSTRICH attribute)}

\begin{fulllineitems}
\phantomsection\label{\detokenize{MouseReferenceManual:Modules.Module_OSTRICH.Module_OSTRICH.optimizer}}\pysigline{\sphinxbfcode{optimizer}}
\sphinxstyleemphasis{str} -- optimization algorithm to use. List can be found in OSTRICH Documentation

\end{fulllineitems}

\index{formatOutput() (Modules.Module\_OSTRICH.Module\_OSTRICH method)}

\begin{fulllineitems}
\phantomsection\label{\detokenize{MouseReferenceManual:Modules.Module_OSTRICH.Module_OSTRICH.formatOutput}}\pysiglinewithargsret{\sphinxbfcode{formatOutput}}{}{}
Formats parameter estimation data for MOUSE
\begin{quote}\begin{description}
\item[{Returns}] \leavevmode
Copies OSTRICH output files to output directory

\item[{Return type}] \leavevmode
None

\end{description}\end{quote}

\end{fulllineitems}

\index{getBoundaryDisplacements() (Modules.Module\_OSTRICH.Module\_OSTRICH method)}

\begin{fulllineitems}
\phantomsection\label{\detokenize{MouseReferenceManual:Modules.Module_OSTRICH.Module_OSTRICH.getBoundaryDisplacements}}\pysiglinewithargsret{\sphinxbfcode{getBoundaryDisplacements}}{}{}
\end{fulllineitems}

\index{getBoundaryStresses() (Modules.Module\_OSTRICH.Module\_OSTRICH method)}

\begin{fulllineitems}
\phantomsection\label{\detokenize{MouseReferenceManual:Modules.Module_OSTRICH.Module_OSTRICH.getBoundaryStresses}}\pysiglinewithargsret{\sphinxbfcode{getBoundaryStresses}}{}{}
\end{fulllineitems}

\index{getModelConstants() (Modules.Module\_OSTRICH.Module\_OSTRICH method)}

\begin{fulllineitems}
\phantomsection\label{\detokenize{MouseReferenceManual:Modules.Module_OSTRICH.Module_OSTRICH.getModelConstants}}\pysiglinewithargsret{\sphinxbfcode{getModelConstants}}{}{}
\end{fulllineitems}

\index{getModelParameters() (Modules.Module\_OSTRICH.Module\_OSTRICH method)}

\begin{fulllineitems}
\phantomsection\label{\detokenize{MouseReferenceManual:Modules.Module_OSTRICH.Module_OSTRICH.getModelParameters}}\pysiglinewithargsret{\sphinxbfcode{getModelParameters}}{}{}
Returns abaqus model input parameters

\begin{sphinxadmonition}{note}{Todo}

move towards a  more object oriented method of handling data
\end{sphinxadmonition}
\begin{quote}\begin{description}
\item[{Returns}] \leavevmode
dictionary of abaqus input parameters

\item[{Return type}] \leavevmode
dict

\end{description}\end{quote}

\end{fulllineitems}

\index{getOstrichParameters() (Modules.Module\_OSTRICH.Module\_OSTRICH method)}

\begin{fulllineitems}
\phantomsection\label{\detokenize{MouseReferenceManual:Modules.Module_OSTRICH.Module_OSTRICH.getOstrichParameters}}\pysiglinewithargsret{\sphinxbfcode{getOstrichParameters}}{\emph{frontBias=1}}{}
Returns OSTRICH parameters

\begin{sphinxadmonition}{note}{Todo}

move towards a  more object oriented method of handling data
\end{sphinxadmonition}
\begin{quote}\begin{description}
\item[{Returns}] \leavevmode
dictionary of udec parameters

\item[{Return type}] \leavevmode
dict

\end{description}\end{quote}

\end{fulllineitems}

\index{parseInput() (Modules.Module\_OSTRICH.Module\_OSTRICH method)}

\begin{fulllineitems}
\phantomsection\label{\detokenize{MouseReferenceManual:Modules.Module_OSTRICH.Module_OSTRICH.parseInput}}\pysiglinewithargsret{\sphinxbfcode{parseInput}}{}{}
Parses input file
\begin{quote}\begin{description}
\item[{Returns}] \leavevmode
creates Ostrich input files

\item[{Return type}] \leavevmode
None

\end{description}\end{quote}

\end{fulllineitems}

\index{run() (Modules.Module\_OSTRICH.Module\_OSTRICH method)}

\begin{fulllineitems}
\phantomsection\label{\detokenize{MouseReferenceManual:Modules.Module_OSTRICH.Module_OSTRICH.run}}\pysiglinewithargsret{\sphinxbfcode{run}}{}{}
runs the OSTRICH Module which creates input files for OSTRICH, then runs OSTRICH
\begin{quote}\begin{description}
\item[{Returns}] \leavevmode
OSTRICH data files

\item[{Return type}] \leavevmode
None

\end{description}\end{quote}

\end{fulllineitems}

\index{setParameters() (Modules.Module\_OSTRICH.Module\_OSTRICH method)}

\begin{fulllineitems}
\phantomsection\label{\detokenize{MouseReferenceManual:Modules.Module_OSTRICH.Module_OSTRICH.setParameters}}\pysiglinewithargsret{\sphinxbfcode{setParameters}}{\emph{args}}{}
Sets module parameters
\begin{quote}\begin{description}
\item[{Parameters}] \leavevmode
\sphinxstyleliteralstrong{parameters} (\sphinxstyleliteralemphasis{dict}) -- new parameters to be set

\item[{Returns}] \leavevmode
Sets module parameters

\item[{Return type}] \leavevmode
None

\end{description}\end{quote}

\end{fulllineitems}


\end{fulllineitems}

\index{fillTemplate() (in module Modules.Module\_OSTRICH)}

\begin{fulllineitems}
\phantomsection\label{\detokenize{MouseReferenceManual:Modules.Module_OSTRICH.fillTemplate}}\pysiglinewithargsret{\sphinxcode{Modules.Module\_OSTRICH.}\sphinxbfcode{fillTemplate}}{\emph{template}, \emph{parameters}, \emph{file}}{}
fills a template file with variable parameters
\begin{quote}\begin{description}
\item[{Parameters}] \leavevmode\begin{itemize}
\item {} 
\sphinxstyleliteralstrong{template} (\sphinxstyleliteralemphasis{str}) -- file path to template file

\item {} 
\sphinxstyleliteralstrong{parameters} (\sphinxstyleliteralemphasis{dict}) -- dictionary of parameters and corresponding values

\item {} 
\sphinxstyleliteralstrong{file} (\sphinxstyleliteralemphasis{str}) -- destination file path for filled template

\end{itemize}

\item[{Returns}] \leavevmode
saves filled template to file

\item[{Return type}] \leavevmode
None

\end{description}\end{quote}

\end{fulllineitems}

\index{getVelocityString() (in module Modules.Module\_OSTRICH)}

\begin{fulllineitems}
\phantomsection\label{\detokenize{MouseReferenceManual:Modules.Module_OSTRICH.getVelocityString}}\pysiglinewithargsret{\sphinxcode{Modules.Module\_OSTRICH.}\sphinxbfcode{getVelocityString}}{\emph{velTable}}{}
Generates a table of relative velocities (from -1 to 1) for the simulation in a linear string format for ABAQUS
\begin{quote}\begin{description}
\item[{Parameters}] \leavevmode
\sphinxstyleliteralstrong{velTable} (\sphinxstyleliteralemphasis{list of float}) -- times at which the velocity changes from negative to positive

\item[{Returns}] \leavevmode
table of relative velocities (-1 to 1) in ABAQUS format

\item[{Return type}] \leavevmode
str

\end{description}\end{quote}

\end{fulllineitems}

\index{importMaterialData() (in module Modules.Module\_OSTRICH)}

\begin{fulllineitems}
\phantomsection\label{\detokenize{MouseReferenceManual:Modules.Module_OSTRICH.importMaterialData}}\pysiglinewithargsret{\sphinxcode{Modules.Module\_OSTRICH.}\sphinxbfcode{importMaterialData}}{\emph{materialName}}{}
Imports the material parameters and assigns them to a global material variable
\begin{quote}\begin{description}
\item[{Parameters}] \leavevmode
\sphinxstyleliteralstrong{material} (\sphinxstyleliteralemphasis{str}) -- Name of file containing the model data.

\item[{Returns}] \leavevmode
Assigns model parameters from file to global material

\item[{Return type}] \leavevmode
None

\end{description}\end{quote}

\end{fulllineitems}

\index{importModelData() (in module Modules.Module\_OSTRICH)}

\begin{fulllineitems}
\phantomsection\label{\detokenize{MouseReferenceManual:Modules.Module_OSTRICH.importModelData}}\pysiglinewithargsret{\sphinxcode{Modules.Module\_OSTRICH.}\sphinxbfcode{importModelData}}{\emph{modelName}}{}
Imports the input model parameters and assigns them to a global modelData variable
\begin{quote}\begin{description}
\item[{Parameters}] \leavevmode
\sphinxstyleliteralstrong{modelName} (\sphinxstyleliteralemphasis{str}) -- Name of file containing the model data.

\item[{Returns}] \leavevmode
Assigns model parameters from file to global modelData

\item[{Return type}] \leavevmode
None

\end{description}\end{quote}

\end{fulllineitems}

\index{parserHandler() (in module Modules.Module\_OSTRICH)}

\begin{fulllineitems}
\phantomsection\label{\detokenize{MouseReferenceManual:Modules.Module_OSTRICH.parserHandler}}\pysiglinewithargsret{\sphinxcode{Modules.Module\_OSTRICH.}\sphinxbfcode{parserHandler}}{\emph{args}}{}
Function called after argparse subparser is executed
\begin{quote}\begin{description}
\item[{Parameters}] \leavevmode
\sphinxstyleliteralstrong{args} (\sphinxstyleliteralemphasis{argparse.Arguments}) -- arparse parsed command line arguments.

\item[{Returns}] \leavevmode
initializes UDEC Module and runs it

\item[{Return type}] \leavevmode
None

\end{description}\end{quote}

\end{fulllineitems}

\index{populateArgumentParser() (in module Modules.Module\_OSTRICH)}

\begin{fulllineitems}
\phantomsection\label{\detokenize{MouseReferenceManual:Modules.Module_OSTRICH.populateArgumentParser}}\pysiglinewithargsret{\sphinxcode{Modules.Module\_OSTRICH.}\sphinxbfcode{populateArgumentParser}}{\emph{parser}}{}
Adds arguments to the argument parser
\begin{quote}\begin{description}
\item[{Parameters}] \leavevmode
\sphinxstyleliteralstrong{parser} (\sphinxstyleliteralemphasis{argparse.ArgumentParser}) -- empty argparse subparser

\item[{Returns}] \leavevmode
same argparse supparser, now populated with arguments

\item[{Return type}] \leavevmode
argparse.ArgumentParser

\end{description}\end{quote}

\end{fulllineitems}



\subsection{Modules.Module\_UDEC module}
\label{\detokenize{MouseReferenceManual:modules-module-udec-module}}\label{\detokenize{MouseReferenceManual:module-Modules.Module_UDEC}}\index{Modules.Module\_UDEC (module)}\index{Module\_UDEC (class in Modules.Module\_UDEC)}

\begin{fulllineitems}
\phantomsection\label{\detokenize{MouseReferenceManual:Modules.Module_UDEC.Module_UDEC}}\pysiglinewithargsret{\sphinxstrong{class }\sphinxcode{Modules.Module\_UDEC.}\sphinxbfcode{Module\_UDEC}}{\emph{baseName}}{}
Bases: {\hyperref[\detokenize{MouseReferenceManual:Modules.Base.DemModuleBaseClass}]{\sphinxcrossref{\sphinxcode{Modules.Base.DemModuleBaseClass}}}}

Creates the UDEC model interface for MOUSE

This class allows for the generation of the usage of UDEC through the MOUSE framework. As of current, there is still no capacity for full automation due to UDEC API limitations. As such, a batch UDEC script is generated which can then be called in UDEC with one command.
\index{fileName (Modules.Module\_UDEC.Module\_UDEC attribute)}

\begin{fulllineitems}
\phantomsection\label{\detokenize{MouseReferenceManual:Modules.Module_UDEC.Module_UDEC.fileName}}\pysigline{\sphinxbfcode{fileName}}
\sphinxstyleemphasis{str} -- name of simulation data file

\end{fulllineitems}

\index{UDECParameters (Modules.Module\_UDEC.Module\_UDEC attribute)}

\begin{fulllineitems}
\phantomsection\label{\detokenize{MouseReferenceManual:Modules.Module_UDEC.Module_UDEC.UDECParameters}}\pysigline{\sphinxbfcode{UDECParameters}}
\sphinxstyleemphasis{dict} -- Dictionary of UDEC parameters as keys and the associated value as dictionary values

\end{fulllineitems}

\index{createInputFiles() (Modules.Module\_UDEC.Module\_UDEC method)}

\begin{fulllineitems}
\phantomsection\label{\detokenize{MouseReferenceManual:Modules.Module_UDEC.Module_UDEC.createInputFiles}}\pysiglinewithargsret{\sphinxbfcode{createInputFiles}}{}{}
Creates Input files for UDEC and a batch file to run them all

\begin{sphinxadmonition}{note}{Todo}

move towards a  more object oriented method of handling data
\end{sphinxadmonition}
\begin{quote}\begin{description}
\item[{Returns}] \leavevmode
creates UDEC input files and corresponding batch file

\item[{Return type}] \leavevmode
None

\end{description}\end{quote}

\end{fulllineitems}

\index{formatOutput() (Modules.Module\_UDEC.Module\_UDEC method)}

\begin{fulllineitems}
\phantomsection\label{\detokenize{MouseReferenceManual:Modules.Module_UDEC.Module_UDEC.formatOutput}}\pysiglinewithargsret{\sphinxbfcode{formatOutput}}{}{}
Formats DEM data into consistent nested hash tables and writes them to binary file
\begin{quote}\begin{description}
\item[{Returns}] \leavevmode
writes serialized binary data to file

\item[{Return type}] \leavevmode
None

\end{description}\end{quote}

\end{fulllineitems}

\index{getUDECParameters() (Modules.Module\_UDEC.Module\_UDEC method)}

\begin{fulllineitems}
\phantomsection\label{\detokenize{MouseReferenceManual:Modules.Module_UDEC.Module_UDEC.getUDECParameters}}\pysiglinewithargsret{\sphinxbfcode{getUDECParameters}}{}{}
Returns UDEC parameters

\begin{sphinxadmonition}{note}{Todo}

move towards a  more object oriented method of handling data
\end{sphinxadmonition}
\begin{quote}\begin{description}
\item[{Returns}] \leavevmode
dictionary of udec parameters

\item[{Return type}] \leavevmode
dict

\end{description}\end{quote}

\end{fulllineitems}

\index{getVelocityString() (Modules.Module\_UDEC.Module\_UDEC method)}

\begin{fulllineitems}
\phantomsection\label{\detokenize{MouseReferenceManual:Modules.Module_UDEC.Module_UDEC.getVelocityString}}\pysiglinewithargsret{\sphinxbfcode{getVelocityString}}{\emph{velTable}}{}
Generates a table of relative velocities (from -1 to 1) for the simulation in a linear string format for UDEC
\begin{quote}\begin{description}
\item[{Parameters}] \leavevmode
\sphinxstyleliteralstrong{velTable} (\sphinxstyleliteralemphasis{list of float}) -- times at which the velocity changes from negative to positive

\item[{Returns}] \leavevmode
table of relative velocities (-1 to 1) in UDEC format

\item[{Return type}] \leavevmode
str

\end{description}\end{quote}

\end{fulllineitems}

\index{inputFileName() (Modules.Module\_UDEC.Module\_UDEC method)}

\begin{fulllineitems}
\phantomsection\label{\detokenize{MouseReferenceManual:Modules.Module_UDEC.Module_UDEC.inputFileName}}\pysiglinewithargsret{\sphinxbfcode{inputFileName}}{}{}
Overloads the default input settings to import python data
\begin{quote}\begin{description}
\item[{Returns}] \leavevmode
full path of input python data

\item[{Return type}] \leavevmode
str

\end{description}\end{quote}

\end{fulllineitems}

\index{outputFileName() (Modules.Module\_UDEC.Module\_UDEC method)}

\begin{fulllineitems}
\phantomsection\label{\detokenize{MouseReferenceManual:Modules.Module_UDEC.Module_UDEC.outputFileName}}\pysiglinewithargsret{\sphinxbfcode{outputFileName}}{}{}
Overloads the default input settings to export
\begin{quote}\begin{description}
\item[{Returns}] \leavevmode
writes serialized binary data to file

\item[{Return type}] \leavevmode
None

\end{description}\end{quote}

\end{fulllineitems}

\index{parseInput() (Modules.Module\_UDEC.Module\_UDEC method)}

\begin{fulllineitems}
\phantomsection\label{\detokenize{MouseReferenceManual:Modules.Module_UDEC.Module_UDEC.parseInput}}\pysiglinewithargsret{\sphinxbfcode{parseInput}}{}{}
Parses input file
\begin{quote}\begin{description}
\item[{Returns}] \leavevmode
returns data in a structured array

\item[{Return type}] \leavevmode
struct

\end{description}\end{quote}

\end{fulllineitems}

\index{run() (Modules.Module\_UDEC.Module\_UDEC method)}

\begin{fulllineitems}
\phantomsection\label{\detokenize{MouseReferenceManual:Modules.Module_UDEC.Module_UDEC.run}}\pysiglinewithargsret{\sphinxbfcode{run}}{}{}
runs the UDEC Module which creates input files for UDEC, then opens UDEC to allow the user to run UDEC, then compiles the UDEC output to MOUSE compatible output.
\begin{quote}\begin{description}
\item[{Returns}] \leavevmode
MOUSE DEM data files

\item[{Return type}] \leavevmode
None

\end{description}\end{quote}

\end{fulllineitems}

\index{setParameters() (Modules.Module\_UDEC.Module\_UDEC method)}

\begin{fulllineitems}
\phantomsection\label{\detokenize{MouseReferenceManual:Modules.Module_UDEC.Module_UDEC.setParameters}}\pysiglinewithargsret{\sphinxbfcode{setParameters}}{}{}
Sets module parameters
\begin{quote}\begin{description}
\item[{Parameters}] \leavevmode
\sphinxstyleliteralstrong{parameters} (\sphinxstyleliteralemphasis{dict}) -- new parameters to be set

\item[{Returns}] \leavevmode
Sets module parameters

\item[{Return type}] \leavevmode
None

\end{description}\end{quote}

\end{fulllineitems}


\end{fulllineitems}

\index{compileFiles() (in module Modules.Module\_UDEC)}

\begin{fulllineitems}
\phantomsection\label{\detokenize{MouseReferenceManual:Modules.Module_UDEC.compileFiles}}\pysiglinewithargsret{\sphinxcode{Modules.Module\_UDEC.}\sphinxbfcode{compileFiles}}{\emph{simulations}, \emph{files}, \emph{rawPath}, \emph{compiledPath}}{}
compiles the raw UDEC data files for each timestep into one file
\begin{quote}\begin{description}
\item[{Parameters}] \leavevmode\begin{itemize}
\item {} 
\sphinxstyleliteralstrong{simulations} (\sphinxstyleliteralemphasis{list of str}) -- List of simulations names

\item {} 
\sphinxstyleliteralstrong{files} (\sphinxstyleliteralemphasis{list of str}) -- List of files to be compiled

\item {} 
\sphinxstyleliteralstrong{rawPath} (\sphinxstyleliteralemphasis{str}) -- directory in which the raw UDEC data is contained

\item {} 
\sphinxstyleliteralstrong{compiledPath} (\sphinxstyleliteralemphasis{str}) -- directory in which the compiled UDEC data is contained

\end{itemize}

\item[{Returns}] \leavevmode
saves compiled data to file.

\item[{Return type}] \leavevmode
None

\end{description}\end{quote}

\end{fulllineitems}

\index{fileList() (in module Modules.Module\_UDEC)}

\begin{fulllineitems}
\phantomsection\label{\detokenize{MouseReferenceManual:Modules.Module_UDEC.fileList}}\pysiglinewithargsret{\sphinxcode{Modules.Module\_UDEC.}\sphinxbfcode{fileList}}{\emph{path}}{}
Gets list of all files in a given directory
\begin{quote}\begin{description}
\item[{Parameters}] \leavevmode
\sphinxstyleliteralstrong{path} (\sphinxstyleliteralemphasis{str}) -- directory to get file list from

\item[{Returns}] \leavevmode
List of file names in directory

\item[{Return type}] \leavevmode
list of str

\end{description}\end{quote}

\end{fulllineitems}

\index{importModelData() (in module Modules.Module\_UDEC)}

\begin{fulllineitems}
\phantomsection\label{\detokenize{MouseReferenceManual:Modules.Module_UDEC.importModelData}}\pysiglinewithargsret{\sphinxcode{Modules.Module\_UDEC.}\sphinxbfcode{importModelData}}{\emph{modelName}}{}
Imports the input model parameters and assigns them to a global modelData variable
\begin{quote}\begin{description}
\item[{Parameters}] \leavevmode
\sphinxstyleliteralstrong{modelName} (\sphinxstyleliteralemphasis{str}) -- Name of file containing the model data.

\item[{Returns}] \leavevmode
Assigns model parameters from file to global modelData

\item[{Return type}] \leavevmode
None

\end{description}\end{quote}

\end{fulllineitems}

\index{parseDataFile() (in module Modules.Module\_UDEC)}

\begin{fulllineitems}
\phantomsection\label{\detokenize{MouseReferenceManual:Modules.Module_UDEC.parseDataFile}}\pysiglinewithargsret{\sphinxcode{Modules.Module\_UDEC.}\sphinxbfcode{parseDataFile}}{\emph{fileName}}{}
Parses raw DEM data from tab delimited text tile into nested python dictionary

The raw DEM Data is considered to be comprised of six distinct types: block data, contact data, corner data, domain data, grid point data, and zone data. Here, each block, contact, corner, domain, grid point, and zone is assigned a unique 7-digit numeric identifier (assuming here that the number of components in the system does not exceed 10 million) by which the associated data can be accessed. The same identifier may be repeated for different data types.Each DEM data hash table has three levels of nesting. The first level keys are the simulation times, which returns the second level of hash tables. The second level keys are the component identifiers, which returns a third level hash table. In this third level, the component attributes can be accessed using the attribute name as the key.
\begin{quote}\begin{description}
\item[{Parameters}] \leavevmode
\sphinxstyleliteralstrong{fileName} (\sphinxstyleliteralemphasis{str}) -- name of data file to be parsed

\item[{Returns}] \leavevmode
Tripple nested dictionary of DEM data

\item[{Return type}] \leavevmode
nested DEM dict

\end{description}\end{quote}

\end{fulllineitems}

\index{parserHandler() (in module Modules.Module\_UDEC)}

\begin{fulllineitems}
\phantomsection\label{\detokenize{MouseReferenceManual:Modules.Module_UDEC.parserHandler}}\pysiglinewithargsret{\sphinxcode{Modules.Module\_UDEC.}\sphinxbfcode{parserHandler}}{\emph{args}}{}
Function called after argparse subparser is executed
\begin{quote}\begin{description}
\item[{Parameters}] \leavevmode
\sphinxstyleliteralstrong{args} (\sphinxstyleliteralemphasis{argparse.Arguments}) -- arparse parsed command line arguments.

\item[{Returns}] \leavevmode
initializes UDEC Module and runs it

\item[{Return type}] \leavevmode
None

\end{description}\end{quote}

\end{fulllineitems}

\index{populateArgumentParser() (in module Modules.Module\_UDEC)}

\begin{fulllineitems}
\phantomsection\label{\detokenize{MouseReferenceManual:Modules.Module_UDEC.populateArgumentParser}}\pysiglinewithargsret{\sphinxcode{Modules.Module\_UDEC.}\sphinxbfcode{populateArgumentParser}}{\emph{parser}}{}
Adds arguments to the argument parser
\begin{quote}\begin{description}
\item[{Parameters}] \leavevmode
\sphinxstyleliteralstrong{parser} (\sphinxstyleliteralemphasis{argparse.ArgumentParser}) -- empty argparse subparser

\item[{Returns}] \leavevmode
same argparse supparser, now populated with arguments

\item[{Return type}] \leavevmode
argparse.ArgumentParser

\end{description}\end{quote}

\end{fulllineitems}

\index{simulationFiles() (in module Modules.Module\_UDEC)}

\begin{fulllineitems}
\phantomsection\label{\detokenize{MouseReferenceManual:Modules.Module_UDEC.simulationFiles}}\pysiglinewithargsret{\sphinxcode{Modules.Module\_UDEC.}\sphinxbfcode{simulationFiles}}{\emph{files}, \emph{rawPath}, \emph{compiledPath}}{}
Isolates files that contain UDEC simulation Data
\begin{quote}\begin{description}
\item[{Parameters}] \leavevmode\begin{itemize}
\item {} 
\sphinxstyleliteralstrong{files} (\sphinxstyleliteralemphasis{list of str}) -- List of files to be searched

\item {} 
\sphinxstyleliteralstrong{rawPath} (\sphinxstyleliteralemphasis{str}) -- directory in which the raw UDEC data is contained

\item {} 
\sphinxstyleliteralstrong{compiledPath} (\sphinxstyleliteralemphasis{str}) -- directory in which the compiled UDEC data is contained

\end{itemize}

\item[{Returns}] \leavevmode
List of simulation files without file extension

\item[{Return type}] \leavevmode
list of str

\end{description}\end{quote}

\end{fulllineitems}



\subsection{Modules.Base module}
\label{\detokenize{MouseReferenceManual:modules-base-module}}\label{\detokenize{MouseReferenceManual:module-Modules.Base}}\index{Modules.Base (module)}\index{ContinuumModuleBaseClass (class in Modules.Base)}

\begin{fulllineitems}
\phantomsection\label{\detokenize{MouseReferenceManual:Modules.Base.ContinuumModuleBaseClass}}\pysiglinewithargsret{\sphinxstrong{class }\sphinxcode{Modules.Base.}\sphinxbfcode{ContinuumModuleBaseClass}}{\emph{program}, \emph{parameters}, \emph{baseName}}{}
Bases: {\hyperref[\detokenize{MouseReferenceManual:Modules.Base.ModuleBaseClass}]{\sphinxcrossref{\sphinxcode{Modules.Base.ModuleBaseClass}}}}

Creates a base class for the continuum model modules containing common methods and attributes

A base continuum model module class is implemented here, inheriting from the module base class to provide a framework containing required methods and attributes for the continuum model modules to inherit. The module class contains methods pertaining to I/O routines associated with the module so that each module that is written behaves in a consistent manner and to avoid reimplementation of certain methods.
\index{type (Modules.Base.ContinuumModuleBaseClass attribute)}

\begin{fulllineitems}
\phantomsection\label{\detokenize{MouseReferenceManual:Modules.Base.ContinuumModuleBaseClass.type}}\pysigline{\sphinxbfcode{type}}
\sphinxstyleemphasis{str} -- Type of module

\end{fulllineitems}

\index{inputFileName() (Modules.Base.ContinuumModuleBaseClass method)}

\begin{fulllineitems}
\phantomsection\label{\detokenize{MouseReferenceManual:Modules.Base.ContinuumModuleBaseClass.inputFileName}}\pysiglinewithargsret{\sphinxbfcode{inputFileName}}{}{}
Returns full path of input binary data
\begin{quote}\begin{description}
\item[{Returns}] \leavevmode
full path of input binary data

\item[{Return type}] \leavevmode
str

\end{description}\end{quote}

\end{fulllineitems}

\index{outputFileName() (Modules.Base.ContinuumModuleBaseClass method)}

\begin{fulllineitems}
\phantomsection\label{\detokenize{MouseReferenceManual:Modules.Base.ContinuumModuleBaseClass.outputFileName}}\pysiglinewithargsret{\sphinxbfcode{outputFileName}}{}{}
Returns full path of output binary data
\begin{quote}\begin{description}
\item[{Returns}] \leavevmode
full path of output binary data

\item[{Return type}] \leavevmode
str

\end{description}\end{quote}

\end{fulllineitems}


\end{fulllineitems}

\index{DemModuleBaseClass (class in Modules.Base)}

\begin{fulllineitems}
\phantomsection\label{\detokenize{MouseReferenceManual:Modules.Base.DemModuleBaseClass}}\pysiglinewithargsret{\sphinxstrong{class }\sphinxcode{Modules.Base.}\sphinxbfcode{DemModuleBaseClass}}{\emph{program}, \emph{parameters}, \emph{baseName}}{}
Bases: {\hyperref[\detokenize{MouseReferenceManual:Modules.Base.ModuleBaseClass}]{\sphinxcrossref{\sphinxcode{Modules.Base.ModuleBaseClass}}}}

Creates a base class for the DEM modules containing common methods and attributes

A base dem module class is implemented here, inheriting from the module base class to provide a framework containing required methods and attributes for the DEM modules to inherit. The module class contains methods pertaining to I/O routines associated with the module so that each module that is written behaves in a consistent manner and to avoid reimplementation of certain methods.
\index{type (Modules.Base.DemModuleBaseClass attribute)}

\begin{fulllineitems}
\phantomsection\label{\detokenize{MouseReferenceManual:Modules.Base.DemModuleBaseClass.type}}\pysigline{\sphinxbfcode{type}}
\sphinxstyleemphasis{str} -- Type of module

\end{fulllineitems}

\index{inputFileName() (Modules.Base.DemModuleBaseClass method)}

\begin{fulllineitems}
\phantomsection\label{\detokenize{MouseReferenceManual:Modules.Base.DemModuleBaseClass.inputFileName}}\pysiglinewithargsret{\sphinxbfcode{inputFileName}}{}{}
Returns full path of input binary data
\begin{quote}\begin{description}
\item[{Returns}] \leavevmode
full path of input binary data

\item[{Return type}] \leavevmode
str

\end{description}\end{quote}

\end{fulllineitems}

\index{outputFileName() (Modules.Base.DemModuleBaseClass method)}

\begin{fulllineitems}
\phantomsection\label{\detokenize{MouseReferenceManual:Modules.Base.DemModuleBaseClass.outputFileName}}\pysiglinewithargsret{\sphinxbfcode{outputFileName}}{}{}
Returns full path of output binary data
\begin{quote}\begin{description}
\item[{Returns}] \leavevmode
full path of output binary data

\item[{Return type}] \leavevmode
str

\end{description}\end{quote}

\end{fulllineitems}


\end{fulllineitems}

\index{HomogenizationModuleBaseClass (class in Modules.Base)}

\begin{fulllineitems}
\phantomsection\label{\detokenize{MouseReferenceManual:Modules.Base.HomogenizationModuleBaseClass}}\pysiglinewithargsret{\sphinxstrong{class }\sphinxcode{Modules.Base.}\sphinxbfcode{HomogenizationModuleBaseClass}}{\emph{program}, \emph{parameters}, \emph{baseName}}{}
Bases: {\hyperref[\detokenize{MouseReferenceManual:Modules.Base.ModuleBaseClass}]{\sphinxcrossref{\sphinxcode{Modules.Base.ModuleBaseClass}}}}

Creates a base class for the homogenization modules containing common methods and attributes

A base homogenization module class is implemented here, inheriting from the module base class to provide a framework containing required methods and attributes for the homogenization modules to inherit. The module class contains methods pertaining to I/O routines associated with the module so that each module that is written behaves in a consistent manner and to avoid reimplementation of certain methods.
\index{type (Modules.Base.HomogenizationModuleBaseClass attribute)}

\begin{fulllineitems}
\phantomsection\label{\detokenize{MouseReferenceManual:Modules.Base.HomogenizationModuleBaseClass.type}}\pysigline{\sphinxbfcode{type}}
\sphinxstyleemphasis{str} -- Type of module

\end{fulllineitems}

\index{inputFileName() (Modules.Base.HomogenizationModuleBaseClass method)}

\begin{fulllineitems}
\phantomsection\label{\detokenize{MouseReferenceManual:Modules.Base.HomogenizationModuleBaseClass.inputFileName}}\pysiglinewithargsret{\sphinxbfcode{inputFileName}}{}{}
Gets full path of input binary data
\begin{quote}\begin{description}
\item[{Returns}] \leavevmode
full path of input binary data

\item[{Return type}] \leavevmode
str

\end{description}\end{quote}

\end{fulllineitems}

\index{outputFileName() (Modules.Base.HomogenizationModuleBaseClass method)}

\begin{fulllineitems}
\phantomsection\label{\detokenize{MouseReferenceManual:Modules.Base.HomogenizationModuleBaseClass.outputFileName}}\pysiglinewithargsret{\sphinxbfcode{outputFileName}}{}{}
Returns full path of output binary data
\begin{quote}\begin{description}
\item[{Returns}] \leavevmode
full path of output binary data

\item[{Return type}] \leavevmode
str

\end{description}\end{quote}

\end{fulllineitems}


\end{fulllineitems}

\index{ModuleBaseClass (class in Modules.Base)}

\begin{fulllineitems}
\phantomsection\label{\detokenize{MouseReferenceManual:Modules.Base.ModuleBaseClass}}\pysiglinewithargsret{\sphinxstrong{class }\sphinxcode{Modules.Base.}\sphinxbfcode{ModuleBaseClass}}{\emph{program}, \emph{baseName}, \emph{parameters=\{\}}, \emph{suppressText=False}, \emph{suppressErrors=True}}{}
Bases: \sphinxcode{object}

Creates a base class containing common module methods and attributes

A base module class is implemented here to provide a framework containing required methods and attributes for the MOUSE modules to inherit. The module class contains methods pertaining to I/O routines associated with the module so that each module that is written behaves in a consistent manner and to avoid reimplementation of certain methods.
\index{program (Modules.Base.ModuleBaseClass attribute)}

\begin{fulllineitems}
\phantomsection\label{\detokenize{MouseReferenceManual:Modules.Base.ModuleBaseClass.program}}\pysigline{\sphinxbfcode{program}}
\sphinxstyleemphasis{str} -- String containing name of module software executable file.

\end{fulllineitems}

\index{parameters (Modules.Base.ModuleBaseClass attribute)}

\begin{fulllineitems}
\phantomsection\label{\detokenize{MouseReferenceManual:Modules.Base.ModuleBaseClass.parameters}}\pysigline{\sphinxbfcode{parameters}}
\sphinxstyleemphasis{dict} -- Dictionary of command line parameters as keys and corresponding arguments as entries

\end{fulllineitems}

\index{suppressText (Modules.Base.ModuleBaseClass attribute)}

\begin{fulllineitems}
\phantomsection\label{\detokenize{MouseReferenceManual:Modules.Base.ModuleBaseClass.suppressText}}\pysigline{\sphinxbfcode{suppressText}}
\sphinxstyleemphasis{bool} -- Suppreses text output from modules if True

\end{fulllineitems}

\index{suppressErrors (Modules.Base.ModuleBaseClass attribute)}

\begin{fulllineitems}
\phantomsection\label{\detokenize{MouseReferenceManual:Modules.Base.ModuleBaseClass.suppressErrors}}\pysigline{\sphinxbfcode{suppressErrors}}
\sphinxstyleemphasis{bool} -- Suppress error output from modules if True

\end{fulllineitems}

\index{baseName (Modules.Base.ModuleBaseClass attribute)}

\begin{fulllineitems}
\phantomsection\label{\detokenize{MouseReferenceManual:Modules.Base.ModuleBaseClass.baseName}}\pysigline{\sphinxbfcode{baseName}}
\sphinxstyleemphasis{str} -- Name of model input file

\end{fulllineitems}

\index{binaryDirectory (Modules.Base.ModuleBaseClass attribute)}

\begin{fulllineitems}
\phantomsection\label{\detokenize{MouseReferenceManual:Modules.Base.ModuleBaseClass.binaryDirectory}}\pysigline{\sphinxbfcode{binaryDirectory}}
\sphinxstyleemphasis{str} -- Directory in which MOUSE binary data is located

\end{fulllineitems}

\index{textDirectory (Modules.Base.ModuleBaseClass attribute)}

\begin{fulllineitems}
\phantomsection\label{\detokenize{MouseReferenceManual:Modules.Base.ModuleBaseClass.textDirectory}}\pysigline{\sphinxbfcode{textDirectory}}
\sphinxstyleemphasis{str} -- Directory in which MOUSE text data is located

\end{fulllineitems}

\index{inputDirectory (Modules.Base.ModuleBaseClass attribute)}

\begin{fulllineitems}
\phantomsection\label{\detokenize{MouseReferenceManual:Modules.Base.ModuleBaseClass.inputDirectory}}\pysigline{\sphinxbfcode{inputDirectory}}
\sphinxstyleemphasis{str} -- Directory in which MOUSE input data is located

\end{fulllineitems}

\index{outputDirectory (Modules.Base.ModuleBaseClass attribute)}

\begin{fulllineitems}
\phantomsection\label{\detokenize{MouseReferenceManual:Modules.Base.ModuleBaseClass.outputDirectory}}\pysigline{\sphinxbfcode{outputDirectory}}
\sphinxstyleemphasis{str} -- Directory in which MOUSE output data is located

\end{fulllineitems}

\index{clearScreen() (Modules.Base.ModuleBaseClass method)}

\begin{fulllineitems}
\phantomsection\label{\detokenize{MouseReferenceManual:Modules.Base.ModuleBaseClass.clearScreen}}\pysiglinewithargsret{\sphinxbfcode{clearScreen}}{}{}
Clears all text from the console.

Returns:
None: Clears all text from the console

\end{fulllineitems}

\index{commandLineArguments() (Modules.Base.ModuleBaseClass method)}

\begin{fulllineitems}
\phantomsection\label{\detokenize{MouseReferenceManual:Modules.Base.ModuleBaseClass.commandLineArguments}}\pysiglinewithargsret{\sphinxbfcode{commandLineArguments}}{}{}
converts the parameters dictionary to a string which can be passed to the command line when running the specified program.
\begin{quote}\begin{description}
\item[{Returns}] \leavevmode
string to be passed to command line

\item[{Return type}] \leavevmode
str

\end{description}\end{quote}

\end{fulllineitems}

\index{loadData() (Modules.Base.ModuleBaseClass method)}

\begin{fulllineitems}
\phantomsection\label{\detokenize{MouseReferenceManual:Modules.Base.ModuleBaseClass.loadData}}\pysiglinewithargsret{\sphinxbfcode{loadData}}{}{}
Loads module data from binary using the pickle serialization module
\begin{quote}\begin{description}
\item[{Parameters}] \leavevmode
\sphinxstyleliteralstrong{data} (\sphinxstyleliteralemphasis{any}) -- Module data to be serialized and stored in file

\item[{Returns}] \leavevmode
serialized data in binary file in specified binaryDirectory

\item[{Return type}] \leavevmode
None

\end{description}\end{quote}

\end{fulllineitems}

\index{printDone() (Modules.Base.ModuleBaseClass method)}

\begin{fulllineitems}
\phantomsection\label{\detokenize{MouseReferenceManual:Modules.Base.ModuleBaseClass.printDone}}\pysiglinewithargsret{\sphinxbfcode{printDone}}{}{}
Prints `Done' to console.

Note:
It is recommended that this method be used in conjunction with printStatus()
\begin{quote}\begin{description}
\item[{Parameters}] \leavevmode
\sphinxstyleliteralstrong{status} (\sphinxstyleliteralemphasis{str}) -- status to be printed to console

\item[{Returns}] \leavevmode
`Done' printed to the console

\item[{Return type}] \leavevmode
None

\end{description}\end{quote}

\end{fulllineitems}

\index{printErrors() (Modules.Base.ModuleBaseClass method)}

\begin{fulllineitems}
\phantomsection\label{\detokenize{MouseReferenceManual:Modules.Base.ModuleBaseClass.printErrors}}\pysiglinewithargsret{\sphinxbfcode{printErrors}}{\emph{error}}{}
Prints error to console if not suppressed

\begin{sphinxadmonition}{note}{Note:}
All errors caught should be routed through this function. Using this function allows for easy suppression and piping of output.
\end{sphinxadmonition}
\begin{quote}\begin{description}
\item[{Parameters}] \leavevmode
\sphinxstyleliteralstrong{error} (\sphinxstyleliteralemphasis{str}) -- error to be printed to console

\item[{Returns}] \leavevmode
error printed to the console

\item[{Return type}] \leavevmode
None

\end{description}\end{quote}

\end{fulllineitems}

\index{printSection() (Modules.Base.ModuleBaseClass method)}

\begin{fulllineitems}
\phantomsection\label{\detokenize{MouseReferenceManual:Modules.Base.ModuleBaseClass.printSection}}\pysiglinewithargsret{\sphinxbfcode{printSection}}{\emph{section}}{}
Prints a section name to console.

Sections are displayed alligned to the left side of the console.
\begin{quote}\begin{description}
\item[{Parameters}] \leavevmode
\sphinxstyleliteralstrong{section} (\sphinxstyleliteralemphasis{str}) -- section name to be printed to console

\item[{Returns}] \leavevmode
section name printed to the console

\item[{Return type}] \leavevmode
None

\end{description}\end{quote}

\end{fulllineitems}

\index{printStatus() (Modules.Base.ModuleBaseClass method)}

\begin{fulllineitems}
\phantomsection\label{\detokenize{MouseReferenceManual:Modules.Base.ModuleBaseClass.printStatus}}\pysiglinewithargsret{\sphinxbfcode{printStatus}}{\emph{status}}{}
Prints a status to console.

Statuses are displayed proceeding a tab and are follwed by ellipses with no new line character at the end of the print line.

Note:
The no new line character at the end of the print line allows the printDone() method to print `Done' at the end of the ellipses after some arbitrary code execution. It is recommended that these two methods always be used together
\begin{quote}\begin{description}
\item[{Parameters}] \leavevmode
\sphinxstyleliteralstrong{status} (\sphinxstyleliteralemphasis{str}) -- status to be printed to console

\item[{Returns}] \leavevmode
status printed to the console

\item[{Return type}] \leavevmode
None

\end{description}\end{quote}

\end{fulllineitems}

\index{printText() (Modules.Base.ModuleBaseClass method)}

\begin{fulllineitems}
\phantomsection\label{\detokenize{MouseReferenceManual:Modules.Base.ModuleBaseClass.printText}}\pysiglinewithargsret{\sphinxbfcode{printText}}{\emph{text}, \emph{end='\textbackslash{}n'}}{}
Prints text to console if not suppressed

\begin{sphinxadmonition}{note}{Note:}
All text printed to the console should be routed through this function rather than using the built-in print() function. Using this function allows for easy suppression and piping of output.
\end{sphinxadmonition}

\begin{sphinxadmonition}{note}{Todo}

If text suppression is on, route output to file.
\end{sphinxadmonition}
\begin{quote}\begin{description}
\item[{Parameters}] \leavevmode\begin{itemize}
\item {} 
\sphinxstyleliteralstrong{text} (\sphinxstyleliteralemphasis{str}) -- text to be printed to console

\item {} 
\sphinxstyleliteralstrong{end} (\sphinxstyleliteralemphasis{str}\sphinxstyleliteralemphasis{, }\sphinxstyleliteralemphasis{optional}) -- character to be appended to end of print line

\end{itemize}

\item[{Returns}] \leavevmode
text printed to the console

\item[{Return type}] \leavevmode
None

\end{description}\end{quote}

\end{fulllineitems}

\index{printTitle() (Modules.Base.ModuleBaseClass method)}

\begin{fulllineitems}
\phantomsection\label{\detokenize{MouseReferenceManual:Modules.Base.ModuleBaseClass.printTitle}}\pysiglinewithargsret{\sphinxbfcode{printTitle}}{\emph{title}}{}
Prints a title to console.

Titles are displayed with horizontal lines printed above and below the text and are alligned with the left side of the console.
\begin{quote}\begin{description}
\item[{Parameters}] \leavevmode
\sphinxstyleliteralstrong{title} (\sphinxstyleliteralemphasis{str}) -- title to be printed to console

\item[{Returns}] \leavevmode
title printed to the console

\item[{Return type}] \leavevmode
None

\end{description}\end{quote}

\end{fulllineitems}

\index{run() (Modules.Base.ModuleBaseClass method)}

\begin{fulllineitems}
\phantomsection\label{\detokenize{MouseReferenceManual:Modules.Base.ModuleBaseClass.run}}\pysiglinewithargsret{\sphinxbfcode{run}}{}{}
runs specified program with specified parameters
\begin{quote}\begin{description}
\item[{Returns}] \leavevmode
runs specified program with specified parameters

\item[{Return type}] \leavevmode
None

\end{description}\end{quote}

\end{fulllineitems}

\index{saveData() (Modules.Base.ModuleBaseClass method)}

\begin{fulllineitems}
\phantomsection\label{\detokenize{MouseReferenceManual:Modules.Base.ModuleBaseClass.saveData}}\pysiglinewithargsret{\sphinxbfcode{saveData}}{\emph{data}}{}
Saves module data as binary using the pickle serialization module
\begin{quote}\begin{description}
\item[{Parameters}] \leavevmode
\sphinxstyleliteralstrong{data} (\sphinxstyleliteralemphasis{any}) -- Module data to be serialized and stored in file

\item[{Returns}] \leavevmode
serialized data in binary file in specified binaryDirectory

\item[{Return type}] \leavevmode
None

\end{description}\end{quote}

\end{fulllineitems}

\index{updateParameters() (Modules.Base.ModuleBaseClass method)}

\begin{fulllineitems}
\phantomsection\label{\detokenize{MouseReferenceManual:Modules.Base.ModuleBaseClass.updateParameters}}\pysiglinewithargsret{\sphinxbfcode{updateParameters}}{\emph{parameters}}{}
Updates the parameter attribute so that the modul can be run with a different parameter set without being re-instantiated
\begin{quote}\begin{description}
\item[{Parameters}] \leavevmode
\sphinxstyleliteralstrong{parameters} (\sphinxstyleliteralemphasis{dict}) -- dictionary of new parameters

\item[{Returns}] \leavevmode
updates the parameter attribute

\item[{Return type}] \leavevmode
None

\end{description}\end{quote}

\end{fulllineitems}


\end{fulllineitems}

\index{ParameterEstimationModuleBaseClass (class in Modules.Base)}

\begin{fulllineitems}
\phantomsection\label{\detokenize{MouseReferenceManual:Modules.Base.ParameterEstimationModuleBaseClass}}\pysiglinewithargsret{\sphinxstrong{class }\sphinxcode{Modules.Base.}\sphinxbfcode{ParameterEstimationModuleBaseClass}}{\emph{program}, \emph{parameters}, \emph{baseName}}{}
Bases: {\hyperref[\detokenize{MouseReferenceManual:Modules.Base.ModuleBaseClass}]{\sphinxcrossref{\sphinxcode{Modules.Base.ModuleBaseClass}}}}

Creates a base class for the parameter estimation modules containing common methods and attributes

A base parameter estimation module class is implemented here, inheriting from the module base class to provide a framework containing required methods and attributes for the parameter estimation modules to inherit. The module class contains methods pertaining to I/O routines associated with the module so that each module that is written behaves in a consistent manner and to avoid reimplementation of certain methods.
\index{type (Modules.Base.ParameterEstimationModuleBaseClass attribute)}

\begin{fulllineitems}
\phantomsection\label{\detokenize{MouseReferenceManual:Modules.Base.ParameterEstimationModuleBaseClass.type}}\pysigline{\sphinxbfcode{type}}
\sphinxstyleemphasis{str} -- Type of module

\end{fulllineitems}

\index{inputFileName() (Modules.Base.ParameterEstimationModuleBaseClass method)}

\begin{fulllineitems}
\phantomsection\label{\detokenize{MouseReferenceManual:Modules.Base.ParameterEstimationModuleBaseClass.inputFileName}}\pysiglinewithargsret{\sphinxbfcode{inputFileName}}{}{}
Returns full path of input binary data
\begin{quote}\begin{description}
\item[{Returns}] \leavevmode
full path of input binary data

\item[{Return type}] \leavevmode
str

\end{description}\end{quote}

\end{fulllineitems}

\index{outputFileName() (Modules.Base.ParameterEstimationModuleBaseClass method)}

\begin{fulllineitems}
\phantomsection\label{\detokenize{MouseReferenceManual:Modules.Base.ParameterEstimationModuleBaseClass.outputFileName}}\pysiglinewithargsret{\sphinxbfcode{outputFileName}}{}{}
Returns full path of output binary data
\begin{quote}\begin{description}
\item[{Returns}] \leavevmode
full path of output binary data

\item[{Return type}] \leavevmode
str

\end{description}\end{quote}

\end{fulllineitems}


\end{fulllineitems}



\chapter{HODS Reference Manual}
\label{\detokenize{HodsReferenceManual::doc}}\label{\detokenize{HodsReferenceManual:hods-reference-manual}}

\section{Modules.HODS package}
\label{\detokenize{HodsReferenceManual:modules-hods-package}}

\subsection{Submodules}
\label{\detokenize{HodsReferenceManual:submodules}}

\subsection{Modules.HODS.HODS module}
\label{\detokenize{HodsReferenceManual:module-Modules.HODS.HODS}}\label{\detokenize{HodsReferenceManual:modules-hods-hods-module}}\index{Modules.HODS.HODS (module)}\index{DataSet (class in Modules.HODS.HODS)}

\begin{fulllineitems}
\phantomsection\label{\detokenize{HodsReferenceManual:Modules.HODS.HODS.DataSet}}\pysiglinewithargsret{\sphinxstrong{class }\sphinxcode{Modules.HODS.HODS.}\sphinxbfcode{DataSet}}{\emph{fileName}}{}
Bases: \sphinxcode{object}
\index{blocksWithContacts() (Modules.HODS.HODS.DataSet method)}

\begin{fulllineitems}
\phantomsection\label{\detokenize{HodsReferenceManual:Modules.HODS.HODS.DataSet.blocksWithContacts}}\pysiglinewithargsret{\sphinxbfcode{blocksWithContacts}}{\emph{blocks}, \emph{contacts}}{}
\end{fulllineitems}

\index{blocksWithCorners() (Modules.HODS.HODS.DataSet method)}

\begin{fulllineitems}
\phantomsection\label{\detokenize{HodsReferenceManual:Modules.HODS.HODS.DataSet.blocksWithCorners}}\pysiglinewithargsret{\sphinxbfcode{blocksWithCorners}}{\emph{blocks}, \emph{corners}}{}
\end{fulllineitems}

\index{contactsBetweenBlocks() (Modules.HODS.HODS.DataSet method)}

\begin{fulllineitems}
\phantomsection\label{\detokenize{HodsReferenceManual:Modules.HODS.HODS.DataSet.contactsBetweenBlocks}}\pysiglinewithargsret{\sphinxbfcode{contactsBetweenBlocks}}{\emph{blocks1}, \emph{blocks2}}{}
\end{fulllineitems}

\index{contactsOnBlocks() (Modules.HODS.HODS.DataSet method)}

\begin{fulllineitems}
\phantomsection\label{\detokenize{HodsReferenceManual:Modules.HODS.HODS.DataSet.contactsOnBlocks}}\pysiglinewithargsret{\sphinxbfcode{contactsOnBlocks}}{\emph{blocks}}{}
\end{fulllineitems}

\index{cornerX() (Modules.HODS.HODS.DataSet method)}

\begin{fulllineitems}
\phantomsection\label{\detokenize{HodsReferenceManual:Modules.HODS.HODS.DataSet.cornerX}}\pysiglinewithargsret{\sphinxbfcode{cornerX}}{\emph{corners}, \emph{time}}{}
\end{fulllineitems}

\index{cornerY() (Modules.HODS.HODS.DataSet method)}

\begin{fulllineitems}
\phantomsection\label{\detokenize{HodsReferenceManual:Modules.HODS.HODS.DataSet.cornerY}}\pysiglinewithargsret{\sphinxbfcode{cornerY}}{\emph{corners}, \emph{time}}{}
\end{fulllineitems}

\index{cornersOnBlocks() (Modules.HODS.HODS.DataSet method)}

\begin{fulllineitems}
\phantomsection\label{\detokenize{HodsReferenceManual:Modules.HODS.HODS.DataSet.cornersOnBlocks}}\pysiglinewithargsret{\sphinxbfcode{cornersOnBlocks}}{\emph{blocks}}{}
\end{fulllineitems}

\index{cornersOnContacts() (Modules.HODS.HODS.DataSet method)}

\begin{fulllineitems}
\phantomsection\label{\detokenize{HodsReferenceManual:Modules.HODS.HODS.DataSet.cornersOnContacts}}\pysiglinewithargsret{\sphinxbfcode{cornersOnContacts}}{\emph{contacts}}{}
\end{fulllineitems}

\index{limits() (Modules.HODS.HODS.DataSet method)}

\begin{fulllineitems}
\phantomsection\label{\detokenize{HodsReferenceManual:Modules.HODS.HODS.DataSet.limits}}\pysiglinewithargsret{\sphinxbfcode{limits}}{}{}
\end{fulllineitems}

\index{parseDataFile() (Modules.HODS.HODS.DataSet method)}

\begin{fulllineitems}
\phantomsection\label{\detokenize{HodsReferenceManual:Modules.HODS.HODS.DataSet.parseDataFile}}\pysiglinewithargsret{\sphinxbfcode{parseDataFile}}{\emph{fileName}}{}
\end{fulllineitems}

\index{zoneS11() (Modules.HODS.HODS.DataSet method)}

\begin{fulllineitems}
\phantomsection\label{\detokenize{HodsReferenceManual:Modules.HODS.HODS.DataSet.zoneS11}}\pysiglinewithargsret{\sphinxbfcode{zoneS11}}{\emph{zones}, \emph{time}}{}
\end{fulllineitems}

\index{zoneS12() (Modules.HODS.HODS.DataSet method)}

\begin{fulllineitems}
\phantomsection\label{\detokenize{HodsReferenceManual:Modules.HODS.HODS.DataSet.zoneS12}}\pysiglinewithargsret{\sphinxbfcode{zoneS12}}{\emph{zones}, \emph{time}}{}
\end{fulllineitems}

\index{zoneS22() (Modules.HODS.HODS.DataSet method)}

\begin{fulllineitems}
\phantomsection\label{\detokenize{HodsReferenceManual:Modules.HODS.HODS.DataSet.zoneS22}}\pysiglinewithargsret{\sphinxbfcode{zoneS22}}{\emph{zones}, \emph{time}}{}
\end{fulllineitems}

\index{zoneS33() (Modules.HODS.HODS.DataSet method)}

\begin{fulllineitems}
\phantomsection\label{\detokenize{HodsReferenceManual:Modules.HODS.HODS.DataSet.zoneS33}}\pysiglinewithargsret{\sphinxbfcode{zoneS33}}{\emph{zones}, \emph{time}}{}
\end{fulllineitems}

\index{zonesInBlocks() (Modules.HODS.HODS.DataSet method)}

\begin{fulllineitems}
\phantomsection\label{\detokenize{HodsReferenceManual:Modules.HODS.HODS.DataSet.zonesInBlocks}}\pysiglinewithargsret{\sphinxbfcode{zonesInBlocks}}{\emph{blocks}}{}
\end{fulllineitems}


\end{fulllineitems}

\index{Homogenize (class in Modules.HODS.HODS)}

\begin{fulllineitems}
\phantomsection\label{\detokenize{HodsReferenceManual:Modules.HODS.HODS.Homogenize}}\pysiglinewithargsret{\sphinxstrong{class }\sphinxcode{Modules.HODS.HODS.}\sphinxbfcode{Homogenize}}{\emph{centre}, \emph{radius}, \emph{fileName}}{}
Bases: \sphinxcode{Modules.HODS.HODS.DataSet}
\index{blocksInsideBoundary() (Modules.HODS.HODS.Homogenize method)}

\begin{fulllineitems}
\phantomsection\label{\detokenize{HodsReferenceManual:Modules.HODS.HODS.Homogenize.blocksInsideBoundary}}\pysiglinewithargsret{\sphinxbfcode{blocksInsideBoundary}}{}{}
\end{fulllineitems}

\index{blocksOnBoundary() (Modules.HODS.HODS.Homogenize method)}

\begin{fulllineitems}
\phantomsection\label{\detokenize{HodsReferenceManual:Modules.HODS.HODS.Homogenize.blocksOnBoundary}}\pysiglinewithargsret{\sphinxbfcode{blocksOnBoundary}}{}{}
\end{fulllineitems}

\index{blocksOutsideBoundary() (Modules.HODS.HODS.Homogenize method)}

\begin{fulllineitems}
\phantomsection\label{\detokenize{HodsReferenceManual:Modules.HODS.HODS.Homogenize.blocksOutsideBoundary}}\pysiglinewithargsret{\sphinxbfcode{blocksOutsideBoundary}}{}{}
\end{fulllineitems}

\index{calculateHomogenizationParameters() (Modules.HODS.HODS.Homogenize method)}

\begin{fulllineitems}
\phantomsection\label{\detokenize{HodsReferenceManual:Modules.HODS.HODS.Homogenize.calculateHomogenizationParameters}}\pysiglinewithargsret{\sphinxbfcode{calculateHomogenizationParameters}}{}{}
\end{fulllineitems}

\index{contactsInsideBoundary() (Modules.HODS.HODS.Homogenize method)}

\begin{fulllineitems}
\phantomsection\label{\detokenize{HodsReferenceManual:Modules.HODS.HODS.Homogenize.contactsInsideBoundary}}\pysiglinewithargsret{\sphinxbfcode{contactsInsideBoundary}}{}{}
\end{fulllineitems}

\index{contactsOutsideBoundary() (Modules.HODS.HODS.Homogenize method)}

\begin{fulllineitems}
\phantomsection\label{\detokenize{HodsReferenceManual:Modules.HODS.HODS.Homogenize.contactsOutsideBoundary}}\pysiglinewithargsret{\sphinxbfcode{contactsOutsideBoundary}}{}{}
\end{fulllineitems}

\index{cornersInsideBoundary() (Modules.HODS.HODS.Homogenize method)}

\begin{fulllineitems}
\phantomsection\label{\detokenize{HodsReferenceManual:Modules.HODS.HODS.Homogenize.cornersInsideBoundary}}\pysiglinewithargsret{\sphinxbfcode{cornersInsideBoundary}}{}{}
\end{fulllineitems}

\index{cornersOutsideBoundary() (Modules.HODS.HODS.Homogenize method)}

\begin{fulllineitems}
\phantomsection\label{\detokenize{HodsReferenceManual:Modules.HODS.HODS.Homogenize.cornersOutsideBoundary}}\pysiglinewithargsret{\sphinxbfcode{cornersOutsideBoundary}}{}{}
\end{fulllineitems}

\index{duplicateCorners() (Modules.HODS.HODS.Homogenize method)}

\begin{fulllineitems}
\phantomsection\label{\detokenize{HodsReferenceManual:Modules.HODS.HODS.Homogenize.duplicateCorners}}\pysiglinewithargsret{\sphinxbfcode{duplicateCorners}}{\emph{corners}, \emph{blocks}}{}
\end{fulllineitems}

\index{orderBlocks() (Modules.HODS.HODS.Homogenize method)}

\begin{fulllineitems}
\phantomsection\label{\detokenize{HodsReferenceManual:Modules.HODS.HODS.Homogenize.orderBlocks}}\pysiglinewithargsret{\sphinxbfcode{orderBlocks}}{\emph{blocks}, \emph{relaventContacts}}{}
\end{fulllineitems}

\index{orderCorners() (Modules.HODS.HODS.Homogenize method)}

\begin{fulllineitems}
\phantomsection\label{\detokenize{HodsReferenceManual:Modules.HODS.HODS.Homogenize.orderCorners}}\pysiglinewithargsret{\sphinxbfcode{orderCorners}}{\emph{orderedBlocks}, \emph{corners}}{}
\end{fulllineitems}

\index{singleElementCorners() (Modules.HODS.HODS.Homogenize method)}

\begin{fulllineitems}
\phantomsection\label{\detokenize{HodsReferenceManual:Modules.HODS.HODS.Homogenize.singleElementCorners}}\pysiglinewithargsret{\sphinxbfcode{singleElementCorners}}{}{}
\end{fulllineitems}

\index{strain() (Modules.HODS.HODS.Homogenize method)}

\begin{fulllineitems}
\phantomsection\label{\detokenize{HodsReferenceManual:Modules.HODS.HODS.Homogenize.strain}}\pysiglinewithargsret{\sphinxbfcode{strain}}{}{}
\end{fulllineitems}

\index{stress() (Modules.HODS.HODS.Homogenize method)}

\begin{fulllineitems}
\phantomsection\label{\detokenize{HodsReferenceManual:Modules.HODS.HODS.Homogenize.stress}}\pysiglinewithargsret{\sphinxbfcode{stress}}{}{}
\end{fulllineitems}

\index{time() (Modules.HODS.HODS.Homogenize method)}

\begin{fulllineitems}
\phantomsection\label{\detokenize{HodsReferenceManual:Modules.HODS.HODS.Homogenize.time}}\pysiglinewithargsret{\sphinxbfcode{time}}{}{}
\end{fulllineitems}


\end{fulllineitems}

\index{common (class in Modules.HODS.HODS)}

\begin{fulllineitems}
\phantomsection\label{\detokenize{HodsReferenceManual:Modules.HODS.HODS.common}}\pysigline{\sphinxstrong{class }\sphinxcode{Modules.HODS.HODS.}\sphinxbfcode{common}}
Bases: \sphinxcode{object}
\index{angle() (Modules.HODS.HODS.common method)}

\begin{fulllineitems}
\phantomsection\label{\detokenize{HodsReferenceManual:Modules.HODS.HODS.common.angle}}\pysiglinewithargsret{\sphinxbfcode{angle}}{\emph{x1}, \emph{y1}, \emph{x2}, \emph{y2}}{}
\end{fulllineitems}

\index{area() (Modules.HODS.HODS.common method)}

\begin{fulllineitems}
\phantomsection\label{\detokenize{HodsReferenceManual:Modules.HODS.HODS.common.area}}\pysiglinewithargsret{\sphinxbfcode{area}}{\emph{p}}{}
\end{fulllineitems}

\index{listIntersection() (Modules.HODS.HODS.common method)}

\begin{fulllineitems}
\phantomsection\label{\detokenize{HodsReferenceManual:Modules.HODS.HODS.common.listIntersection}}\pysiglinewithargsret{\sphinxbfcode{listIntersection}}{\emph{a}, \emph{b}}{}
\end{fulllineitems}

\index{segments() (Modules.HODS.HODS.common method)}

\begin{fulllineitems}
\phantomsection\label{\detokenize{HodsReferenceManual:Modules.HODS.HODS.common.segments}}\pysiglinewithargsret{\sphinxbfcode{segments}}{\emph{p}}{}
\end{fulllineitems}

\index{triangleArea() (Modules.HODS.HODS.common method)}

\begin{fulllineitems}
\phantomsection\label{\detokenize{HodsReferenceManual:Modules.HODS.HODS.common.triangleArea}}\pysiglinewithargsret{\sphinxbfcode{triangleArea}}{\emph{gp}}{}
\end{fulllineitems}


\end{fulllineitems}



\subsection{Module contents}
\label{\detokenize{HodsReferenceManual:module-Modules.HODS}}\label{\detokenize{HodsReferenceManual:module-contents}}\index{Modules.HODS (module)}

\chapter{Indices and tables}
\label{\detokenize{index:indices-and-tables}}\begin{itemize}
\item {} 
\DUrole{xref,std,std-ref}{genindex}

\item {} 
\DUrole{xref,std,std-ref}{modindex}

\item {} 
\DUrole{xref,std,std-ref}{search}

\end{itemize}


\renewcommand{\indexname}{Python Module Index}
\begin{sphinxtheindex}
\def\bigletter#1{{\Large\sffamily#1}\nopagebreak\vspace{1mm}}
\bigletter{m}
\item {\sphinxstyleindexentry{Modules.Base}}\sphinxstyleindexpageref{MouseReferenceManual:\detokenize{module-Modules.Base}}
\item {\sphinxstyleindexentry{Modules.Module\_ABAQUS}}\sphinxstyleindexpageref{MouseReferenceManual:\detokenize{module-Modules.Module_ABAQUS}}
\item {\sphinxstyleindexentry{Modules.Module\_HODS}}\sphinxstyleindexpageref{MouseReferenceManual:\detokenize{module-Modules.Module_HODS}}
\item {\sphinxstyleindexentry{Modules.Module\_OSTRICH}}\sphinxstyleindexpageref{MouseReferenceManual:\detokenize{module-Modules.Module_OSTRICH}}
\item {\sphinxstyleindexentry{Modules.Module\_UDEC}}\sphinxstyleindexpageref{MouseReferenceManual:\detokenize{module-Modules.Module_UDEC}}
\item {\sphinxstyleindexentry{MOUSE}}\sphinxstyleindexpageref{MouseReferenceManual:\detokenize{module-MOUSE}}
\end{sphinxtheindex}

\renewcommand{\indexname}{Index}
\printindex
\end{document}